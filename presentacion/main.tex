\documentclass{beamer}

\usepackage[T1]{fontenc}
\usepackage[utf8]{inputenc}
\usepackage{lmodern}
\usepackage{graphicx}
\usepackage{svg}

\usetheme{Warsaw}
\logo{\includegraphics[height=1cm]{dna.png}}

\title[A.C. en la Simulación de la Dinámica de Dominios Proteicos]{Autómatas Celulares en la Simulación de la Dinámica de Dominios Proteicos}
\author[Universidad Nacional Autónoma de México]{Juan C. Castrejón E.}
\institute{Universidad Nacional Autónoma de México}
\date{17 de abril de 2023}

\begin{document}

\begin{frame}
\titlepage
\end{frame}

\begin{frame}{Introducción}
\begin{itemize}
\item Importancia de la evolución de dominios de proteínas en la biología y medicina.
\item Uso de métodos computacionales para modelar y predecir evolución de dominios de proteínas.
\item Automatas celulares como herramienta poderosa para modelar sistemas dinámicos complejos.
\end{itemize}
\end{frame}

\begin{frame}{Pregunta de Investigación}
\begin{itemize}
\item ¿Cómo se pueden utilizar los autómatas celulares para simular la evolución de dominios de proteínas?
\end{itemize}
\end{frame}

\begin{frame}{Objetivo}
\begin{itemize}
\item Desarrollar un modelo basado en autómatas celulares para simular la evolución de dominios de proteínas.
\item Analizar y validar el modelo propuesto utilizando datos experimentales.
\end{itemize}
\end{frame}

\begin{frame}{Métodos (Idea general)}
\begin{columns}
\column{0.5\textwidth}
\begin{itemize}
\item Crear un autómata celular para representar la estructura y función de una proteína.
\item Definir reglas de transición basadas en la dinámica de evolución de dominios de proteínas.
\item Simular la evolución de dominios de proteínas utilizando el autómata celular propuesto.
\end{itemize}
\column{0.5\textwidth}
\begin{figure}[ht]
\centering
\includesvg[width=0.8\textwidth]{CA.svg}
\caption{Ejemplos de reglas.}
\label{fig:direct_svg}
\end{figure}
\end{columns}
\end{frame}

\begin{frame}{Autómatas Celulares}
\begin{figure}[ht]
\centering
\includesvg[width=0.55\textwidth]{CA2.svg}
\caption{Ejemplos de reglas propagadas.}
\label{fig:direct_svg}
\end{figure}
\end{frame}


\begin{frame}{Referencias}
\begin{itemize}
\item Qiu, Y., Zhang, X. (2020). Using Cellular Automata to Simulate Domain Evolution in Proteins. Frontiers in Genetics, 11, 515. \url{https://doi.org/10.3389/fgene.2020.00515}
\end{itemize}
\end{frame}

\begin{frame}{Recursos}
\begin{itemize}
\item Repositorio de Github \url{https://github.com/JCastrejonE/proyecto-computacion-genomica/tree/main/}
\item Código de esta presentación \url{https://github.com/JCastrejonE/proyecto-computacion-genomica/blob/main/presentacion/main.tex}
\end{itemize}
\end{frame}

\end{document}